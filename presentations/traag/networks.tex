\documentclass[hide notes,compress]{beamer}

\usepackage{siunitx}

\usepackage{datatool}

\usepackage[utf8]{inputenc}
\RequirePackage[T1]{fontenc}

\usepackage[
  backend=bibtex,
  hyperref=true,
  url=false,
  isbn=false,
  doi=false,
  eprint=false,
  maxcitenames=3,
  terseinits=true,
  maxbibnames=100,
  style=nature,
  block=none]{biblatex}
\bibliography{bibliography_formatted}
% No title and no month
\AtEveryCitekey{\clearfield{title}}
\AtEveryCitekey{\clearfield{month}}
\AtEveryCitekey{\clearfield{series}}
% Only last name
% This currently produces an error unfortunately
% \DeclareNameFormat{}{\usebibmacro{name:first-last}{}{#5}{#1}{#7}\usebibmacro{name:andothers}}

\usetheme{default}
\useinnertheme{rounded}
\useinnertheme{circles}
\usecolortheme{orchid}
\usepackage{graphicx}
\usepackage{bbm}
\usepackage{amsmath,amssymb}
\usepackage{datatool}
\usepackage{eurosym}
\usepackage[absolute,overlay]{textpos}
\usepackage[normalem]{ulem}

\usepackage{tikz}
\usetikzlibrary{colorbrewer}
\usetikzlibrary{positioning}
\usetikzlibrary{fit}					% fitting shapes to coordinates
\usetikzlibrary{backgrounds}	% drawing the background after the foreground
\usetikzlibrary{matrix}
\usetikzlibrary{patterns}
\usetikzlibrary{shapes,shapes.misc}
\usetikzlibrary{calc,decorations.pathmorphing,patterns}
\usetikzlibrary{arrows}
\usetikzlibrary{calc}
\usetikzlibrary{decorations.pathreplacing}
\usetikzlibrary{fit}					% fitting shapes to coordinates
\usetikzlibrary{topaths}
\usetikzlibrary{spy}
\usetikzlibrary{shadows}
\usetikzlibrary{fadings}
\tikzstyle{every picture}+=[remember picture]

\usepackage{pgfplots}
\usepackage{pgfplotstable}
\usepgfplotslibrary{colorbrewer}
\usepgfplotslibrary{groupplots}
\usepgfplotslibrary{designplots}
\usepgfplotslibrary{dateplot}
\pgfplotsset{compat=1.13}

\newcommand\footnotenonum[1]{%
  \begingroup
  \renewcommand\thefootnote{}\footnote{#1}%
  \addtocounter{footnote}{-1}%
  \endgroup
}
\newrobustcmd*{\footfullcitenomark}{%
  \AtNextCite{%
    \let\thefootnote\relax
    \let\mkbibfootnote\mkbibfootnotetext}%
  \footfullcite}
	
\defbeamertemplate*{footnotetext}{default}{%
  %\parindent 1em\noindent%
  \raggedright
  %\hbox to 1.8em{\hfil\insertfootnotemark}%
  \insertfootnotetext\par}
	
\newrobustcmd*{\footfullcitenomarkleft}{%
  \AtNextCite{%
    \setbeamertemplate{footnote}{\usebeamertemplate{footnotetext}}%
    \let\mkbibfootnote\mkbibfootnotetext}%
  \footfullcite}

\newcommand{\convexpath}[2]{
[   
    create hullnodes/.code={
        \global\edef\namelist{#1}
        \foreach [count=\counter] \nodename in \namelist {
            \global\edef\numberofnodes{\counter}
            \node at (\nodename) [draw=none,name=hullnode\counter] {};
        }
        \node at (hullnode\numberofnodes) [name=hullnode0,draw=none] {};
        \pgfmathtruncatemacro\lastnumber{\numberofnodes+1}
        \node at (hullnode1) [name=hullnode\lastnumber,draw=none] {};
    },
    create hullnodes
]
($(hullnode1)!#2!-90:(hullnode0)$)
\foreach [
    evaluate=\currentnode as \previousnode using \currentnode-1,
    evaluate=\currentnode as \nextnode using \currentnode+1
    ] \currentnode in {1,...,\numberofnodes} {
-- ($(hullnode\currentnode)!#2!-90:(hullnode\previousnode)$)
  let \p1 = ($(hullnode\currentnode)!#2!-90:(hullnode\previousnode) - (hullnode\currentnode)$),
    \n1 = {atan2(\y1,\x1)},
    \p2 = ($(hullnode\currentnode)!#2!90:(hullnode\nextnode) - (hullnode\currentnode)$),
    \n2 = {atan2(\y2,\x2)},
    \n{delta} = {-Mod(\n1-\n2,360)}
  in 
    {arc [start angle=\n1, delta angle=\n{delta}, radius=#2]}
}
-- cycle
}


\tikzset{pencil/.style={
    decorate,
    decoration={random steps,segment length=3mm,amplitude=0.2mm}
  }
}

\tikzset{pencil small/.style={
    decorate,
    decoration={random steps,segment length=2mm,amplitude=0.05mm}
  }
}

\newcommand{\triad}[7]
{
  \node[p] (#1-i) at (210:5)  {#5};
  \node[p] (#1-j) at (330:5)  {#6};
  \node[p] (#1-k) at (90:5)   {#7};
  \path[-, bend left=35]
        (#1-i)	edge [#2,bend right=35] node (#1-ij) {} (#1-j) 
            edge [#3] node (#1-ik) {} (#1-k)
        (#1-k) edge [#4] node (#1-kj) {} (#1-j);
}

\newcommand{\clique}[6]{%
    \foreach \i in {1,...,#2}
    {
      \node[n,minimum size=#6] (#1-\i) at ($(\i/#2*360 + #5:#3)+#4$) {}
	\foreach \j in {1,...,\i} 
	{ edge[e] (#1-\j) }
      ;
    }
}

\newcommand{\cliquealt}[6]{%
    \foreach \i in {1,...,#2}
    {
      \node[n,minimum size=#6] (#1-\i) at ($(\i/#2*360 + #5:#3)+#4$) {}
	\foreach \j in {1,...,\i} 
	{ edge[e\j] (#1-\j) }
      ;
    }
}


\setlength{\TPHorizModule}{1mm}
\setlength{\TPVertModule}{1mm}
%\everymath{\displaystyle}
\newcommand{\one}{\mathbbm{1}}
\newcommand{\T}{{\!\top}}

\pgfdeclarelayer{background} 
\pgfsetlayers{background,main} 

\setlength{\leftmargini}{1em}
\graphicspath{{figures/}}
% Set title and authors
\title{Computation Social Science - Network}
\author{V.A. Traag}
\institute{CWTS, Leiden University}
\titlegraphic{
  \includegraphics[width=2cm]{logo-4tu}
}

%\includeonlyframes{title,toc,overview}

\graphicspath{ {../logos/}{figures/} }
\date{7 April, 2017}
\tikzstyle{every picture}+=[remember picture]
\tikzstyle{na} = [baseline=-.5ex, inner sep=0pt, outer sep=0pt]
\begin{document}
\frame[label=title]{\titlepage}

\begin{frame}[label=overview,c]{Overview}
  \begin{block}{Social network analysis}
    \begin{itemize}
      \item Origin in social sciences.
      \item Starts in $\sim$ 1930s.
      \item Homophily, social influence, weak ties, structural holes,
            equivalence, social balance.
      \item Traditionally observational/survey based, small-scale networks.  
    \end{itemize}
  \end{block}
  \begin{block}{Complex networks}
    \begin{itemize}
      \item Origin in physics/applied mathematics.
      \item Starts end of 20\textsuperscript{th} century.
      \item Random networks, spreading, percolation, community detection.
      \item Large-scale (digital) networks.
    \end{itemize}
  \end{block}
\end{frame}

\section{Network basics}
\begin{frame}[c]
 \begin{center}
   \Huge
   Network basics
 \end{center}
\end{frame}

\tikzstyle{n}=[circle,fill=lightgray,draw=black,
               text=white,font=\tiny,
               minimum size=5mm,
               inner sep=0pt, outer sep=0pt]
\tikzstyle{e}=[gray,opacity=0.5]

\DTLloaddb[noheader=false]{karate_nodes}{data/karate_nodes.csv}
\DTLloaddb[noheader=false]{karate_links}{data/karate_links.csv}

\begin{frame}[t]{Network definition}
  \begin{overlayarea}{\textwidth}{\textheight}
    \centering
    \begin{tikzpicture}
      \draw[transparent,help lines] (-1,-1) grid (10, 7);
      \draw[transparent,thick] (-1, 0) -- (10, 0);
      \draw[transparent,thick] (0, -1) -- (0, 7);
      \begin{scope}[x=8cm,y=5cm]
        % Draw nodes
        \DTLforeach*{karate_nodes}{\n=node, \x=x, \y=y}
        { \node[n] (\n) at (\x,\y) {\n}; }
        % Draw edges 

        \begin{pgfonlayer}{background}
          \DTLforeach*{karate_links}{\u=v1,\v=v2}
          { \draw (\u) edge[e] 
                       node (\u-\v) {}
                  (\v); }
        \end{pgfonlayer}

        \only<2>{
          \node[above left=1cm of 15] (txt_node) {Vertex};
          \draw[->,shorten >=1mm] (txt_node) -- (15);
        }
        \only<3-6>{
          \node[above left=1cm of 15] (txt_node) {Node};
          \draw[->,shorten >=1mm] (txt_node) -- (15);
        }
        \only<4>{
          \node[above right=1cm of 23-25] (txt_link) {Edge};
          \draw[->,shorten >=1mm] (txt_link) -- (23-25);
        }
        \only<5-6>{
          \node[above right=1cm of 23-25] (txt_link) {Link};
          \draw[->,shorten >=1mm] (txt_link) -- (23-25);
        }
        \only<6>{
          \node[above=1.5cm of 23-25,anchor=south,text width=2cm] 
              (txt_weight) {Can be weighted};
          \draw[->,shorten >=1mm] (txt_weight) -- (23-25);
          \draw (23) edge[e,thick] (25);
        }
      \end{scope}
    \end{tikzpicture}  
  \end{overlayarea}
\end{frame}

\begin{frame}[c]{Network definition}

  \begin{columns}
    \begin{column}{.47\textwidth}
      \begin{block}{Nodes}
        \begin{itemize}
          \item People
          \item Organisations
          \item Blogs
          \item Countries
        \end{itemize}
      \end{block}
    \end{column}
    \begin{column}{.47\textwidth}
      \begin{block}{Links}
        \begin{itemize}
          \item Communication
          \item Collaboration
          \item Hyperlinks
          \item Alliances
        \end{itemize}
      \end{block}
    \end{column}
  \end{columns}
  
\end{frame}

\begin{frame}[t]{Degree}
  \begin{overlayarea}{\textwidth}{\textheight}
    \centering
    \begin{tikzpicture}
      \draw[transparent,help lines] (-1,-1) grid (10, 7);
      \draw[transparent,thick] (-1, 0) -- (10, 0);
      \draw[transparent,thick] (0, -1) -- (0, 7);
      \begin{scope}[x=8cm,y=5cm]
        % Draw nodes
        \DTLforeach*{karate_nodes}{\n=node, \x=x, \y=y}
        { \node[n] (\n) at (\x,\y) {\n}; }
        % Draw edges 

        \begin{pgfonlayer}{background}
          \DTLforeach*{karate_links}{\u=v1,\v=v2}
          { \draw (\u) edge[e] 
                       node (\u-\v) {}
                  (\v); }
        \end{pgfonlayer}

        \only<1>{
          \node[above=1cm of 25] (txt_degree) {3 neighbors}; 
          \draw[->,shorten >=1mm] (txt_degree) -- (25);
          \node[n,fill=Set1-3-1] at (25) {25};
          \foreach \v in {23, 31, 24}
          {
            \draw (25) edge[black,thick]
                  (\v);
            \node[n,fill=Set1-3-2] at (\v) {\v};
          }
        }
        \only<2>{
          \node[above=2cm of 32,xshift=-5mm] (txt_degree) {12 neighbors}; 
          \draw[->,shorten >=1mm] (txt_degree) -- (32);
          \node[n,fill=Set1-3-1] at (32) {32};
          \foreach \v in {2,8,14,15,18,20,22,23,29,30,31,33}
          {
            \draw (32) edge[black,thick]
                  (\v);
            \node[n,fill=Set1-3-2] at (\v) {\v};
          }
        }
      \end{scope}
    \end{tikzpicture}  
  \end{overlayarea}
\end{frame}

\begin{frame}[label=paths,t]{Paths}
  \begin{overlayarea}{\textwidth}{\textheight}
    \centering
    \begin{tikzpicture}
      \draw[transparent,help lines] (-1,-1) grid (10, 7);
      \draw[transparent,thick] (-1, 0) -- (10, 0);
      \draw[transparent,thick] (0, -1) -- (0, 7);
      \begin{scope}[x=8cm,y=5cm]
        % Draw nodes
        \DTLforeach*{karate_nodes}{\n=node, \x=x, \y=y}
        { \node[n] (\n) at (\x,\y) {\n}; }
        % Draw edges 

        \begin{pgfonlayer}{background}
          \DTLforeach*{karate_links}{\u=v1,\v=v2}
          { \draw (\u) edge[e] 
                       node (\u-\v) {}
                  (\v); }
        \end{pgfonlayer}

        \only<2>{
          %\node[above=2cm of 32] (txt_degree) {12 neighbors}; 
          %\draw[->,shorten >=1mm] (txt_degree) -- (32);
          %\node[n,fill=Set1-3-1] at (32) {32};
          \foreach \v [remember=\v as \u (initially 16)] 
                      in {16,5,0,2,32,15}
          {
            \draw (\u) edge[Set1-3-1,thick] 
                  (\v);
            \node[n,fill=Set1-3-1] at (\v) {\v};
          }
        }
      \end{scope}
    \end{tikzpicture}  
  \end{overlayarea}
\end{frame}

\DTLloaddb[noheader=false]{karate_components_nodes}{data/karate_components_nodes.csv}
\DTLloaddb[noheader=false]{karate_components_links}{data/karate_components_links.csv}
\begin{frame}[t]{Components}
  \begin{overlayarea}{\textwidth}{\textheight}
    \centering
    \begin{tikzpicture}
      \draw[transparent,help lines] (-1,-1) grid (10, 7);
      \draw[transparent,thick] (-1, 0) -- (10, 0);
      \draw[transparent,thick] (0, -1) -- (0, 7);
      \only<1>{
        \node[anchor=west] at (0, 6) {Delete node 0};
      }
      \only<2>{
        \node[anchor=west] at (0, 6) {Node 0 functioned as a bridge};
      }
      \only<3>{
        \node[anchor=west] at (0, 6) {Usually one \emph{giant component}.};
      }
      \begin{scope}[x=8cm,y=5cm]
        % Draw nodes
        \only<1>{
        \DTLforeach*{karate_nodes}{\n=node, \x=x, \y=y}
        { \node[n] (\n) at (\x,\y) {\n}; }
        % Draw edges 

        \begin{pgfonlayer}{background}
          \DTLforeach*{karate_links}{\u=v1,\v=v2}
          { \draw (\u) edge[e] 
                       node (\u-\v) {}
                  (\v); }
        \end{pgfonlayer}
        }
        \only<2->{
        \DTLforeach*{karate_components_nodes}{\n=node, \x=x, \y=y, \c=cluster, \l=label}
        { \node[n,fill=Set1-3-\c] (\n) at (\x,\y) {\l}; }
        % Draw edges 

        \begin{pgfonlayer}{background}
          \DTLforeach*{karate_components_links}{\u=v1,\v=v2}
          { \draw (\u) edge[e] 
                       node (\u-\v) {}
                  (\v); }
        \end{pgfonlayer}
        }
      \end{scope}
    \end{tikzpicture}  
  \end{overlayarea}
\end{frame}

\begin{frame}[label=cycles,t]{Cycles}
  \begin{overlayarea}{\textwidth}{\textheight}
    \centering
    \begin{tikzpicture}
      \draw[transparent,help lines] (-1,-1) grid (10, 7);
      \draw[transparent,thick] (-1, 0) -- (10, 0);
      \draw[transparent,thick] (0, -1) -- (0, 7);
      \begin{scope}[x=8cm,y=5cm]
        % Draw nodes
        \DTLforeach*{karate_nodes}{\n=node, \x=x, \y=y}
        { \node[n] (\n) at (\x,\y) {\n}; }
        % Draw edges 

        \begin{pgfonlayer}{background}
          \DTLforeach*{karate_links}{\u=v1,\v=v2}
          { \draw (\u) edge[e] 
                       node (\u-\v) {}
                  (\v); }
        \end{pgfonlayer}

        \only<2>{
          %\node[above=2cm of 32] (txt_degree) {12 neighbors}; 
          %\draw[->,shorten >=1mm] (txt_degree) -- (32);
          %\node[n,fill=Set1-3-1] at (32) {32};
          \foreach \v [remember=\v as \u (initially 24)] 
                      in {24,25,23,27,24}
          {
            \draw (\u) edge[Set1-3-1,thick] 
                  (\v);
            \node[n,fill=Set1-3-1] at (\v) {\v};
          }
        }
      \end{scope}
    \end{tikzpicture}  
  \end{overlayarea}
\end{frame}

\DTLloaddb[noheader=false]{karate_ego_nodes}{data/karate_ego_nodes.csv}
\DTLloaddb[noheader=false]{karate_ego_links}{data/karate_ego_links.csv}

\tikzstyle{e1}=[e]
\tikzstyle{e2}=[e,black,opacity=1,thick]
\begin{frame}[t]{Clustering}
  \begin{overlayarea}{\textwidth}{\textheight}
    \centering
    \begin{tikzpicture}
      \draw[transparent,help lines] (-1,-1) grid (10, 7);
      \draw[transparent,thick] (-1, 0) -- (10, 0);
      \draw[transparent,thick] (0, -1) -- (0, 7);
      \only<1>{
        \node[anchor=west] at (0, 6) {Zoom in on node 0};
      }
      \only<2>{
        \node[anchor=west] at (0, 6) {Ego network of node 0};
      }
      \only<3>{
        \node[anchor=west] at (0, 6) {Clustering $ = \frac{18}{120} = 0.15$};
      }
      \begin{scope}[x=8cm,y=5cm]
        % Draw nodes
        \only<1>{
          \DTLforeach*{karate_nodes}{\n=node, \x=x, \y=y}
          { \node[n] (\n) at (\x,\y) {\n}; }
          % Draw edges 

          \begin{pgfonlayer}{background}
            \DTLforeach*{karate_links}{\u=v1,\v=v2}
            { \draw (\u) edge[e] 
                         node (\u-\v) {}
                    (\v); }
          \end{pgfonlayer}

          \node[n,fill=Set1-3-1] at (0) {0};
          \foreach \v in {1, 2, 3, 4, 5, 6, 7, 8, 10, 11, 12, 13, 17, 19, 21, 31}
          {
            \draw (0) edge[black,thick]
                  (\v);
            \node[n,fill=Set1-3-2] at (\v) {\v};
          }
        }
        \only<2->{
        % Draw nodes
        \DTLforeach*{karate_ego_nodes}{\n=node, \x=x, \y=y, \l=label}
        { \node[n] (\n) at (\x,\y) {\l}; }
        % Draw edges 

        \begin{pgfonlayer}{background}
          \DTLforeach*{karate_ego_links}{\u=v1,\v=v2,\t=type}
          { \draw (\u) edge[\t] 
                       node (\u-\v) {}
                  (\v); }
        \end{pgfonlayer}
        }
      \end{scope}
    \end{tikzpicture}  
  \end{overlayarea}
\end{frame}

\section{Social network analysis}
\frame[c]{\Huge \centering Social network analysis}

\subsection{Homophily}
\frame[c]{\Huge \centering Homophily}

\begin{frame}[c]{Homophily}
  \begin{itemize}
    \item ``Birds of a feather flock together''
    \item If people are more alike, they are more likely to be connected.
      \begin{itemize}
        \item Same ethnicity
        \item Same gender
        \item Same music preference
        \item Same education
      \end{itemize}
    \item Can be measured using \emph{assortativity}.
    \item Compares to number of expected edges (configuration model).
    \item For nominal (i.e. categorical) variables or continuous variables.
    \item Degree assortativity: core-periphery.
  \end{itemize}
\end{frame}

\begin{frame}[c]{Homophily}
  Gender assortativity $\approx 0.17$
  \begin{center}
    \includegraphics[width=\linewidth]{sociopatterns_gender}
  \end{center}
\end{frame}

\begin{frame}[c]{Homophily}
  Class assortativity $\approx 0.65$
  \begin{center}
    \includegraphics[width=\linewidth]{sociopatterns_class}
  \end{center}
\end{frame}

\begin{frame}[c]{Homophily}
  Within class gender assortativity $\approx 0.02$
  \begin{center}
    \includegraphics[width=\linewidth]{sociopatterns_class_0_gender}
  \end{center}
\end{frame}


\subsection{Social influence}
\frame[c]{\Huge \centering Social influence}

\begin{frame}[c]{Social influence}

  People that are connected influence each other.

  \vskip1cm

  \begin{columns}
    \begin{column}{.47\textwidth}
      \begin{block}{Simple contagion}
        \begin{itemize}
          \item Infectious diseases
          \item Information cascades
        \end{itemize}
      \end{block}
    \end{column}
    \begin{column}{.47\textwidth}
      \begin{block}{Complex contagion}
        \begin{itemize}
          \item Opinion dynamics
          \item Collective action
        \end{itemize}
      \end{block}
    \end{column}
  \end{columns}

  \vskip1cm

  \begin{alertblock}{Confounding}
    Problem: cannot easily distinguish from homophily!
  \end{alertblock}

\end{frame}

\begin{frame}[label=opinions,t]{Opinions}
  \only<1-6>{Example opinion dynamics, step }%
  \only<1>{1}%
  \only<2>{2}%
  \only<3>{3}%
  \only<4>{4}%
  \only<5>{5}%
  \only<6>{6}
  \only<7>{Opinion assortativity $\approx 0.72$}
  \begin{center}
    \only<1>{\includegraphics[width=\linewidth]{social_influence/sociopatterns_opinion_0}}
    \only<2>{\includegraphics[width=\linewidth]{social_influence/sociopatterns_opinion_1}}
    \only<3>{\includegraphics[width=\linewidth]{social_influence/sociopatterns_opinion_2}}
    \only<4>{\includegraphics[width=\linewidth]{social_influence/sociopatterns_opinion_3}}
    \only<5>{\includegraphics[width=\linewidth]{social_influence/sociopatterns_opinion_4}}
    \only<6->{\includegraphics[width=\linewidth]{social_influence/sociopatterns_opinion_5}}
  \end{center}
\end{frame}


\subsection{Centrality}
\frame[c]{\Huge \centering Centrality}
\begin{frame}[c]{Centrality}
  \begin{itemize}
    \item Importance of a node
      \begin{itemize}
        \item Influence, prestige, status, centrality 
      \end{itemize}
    \item Many possible centralities. 
    \item May be of use in different circumstances.
  \end{itemize}
\end{frame}

\begin{frame}[t]{Degree centrality}
  \begin{overlayarea}{\linewidth}{5mm}
  A node is central if it has many links
  \end{overlayarea}
  \begin{overlayarea}{\linewidth}{\textheight}
    \includegraphics[width=0.9\linewidth]{centralities/karate_degree}
  \end{overlayarea}
\end{frame}

\begin{frame}[t]{Betweenness centrality}
  \begin{overlayarea}{\linewidth}{5mm}
  A node is central if there are many paths passing through
  \end{overlayarea}
  \begin{overlayarea}{\linewidth}{\textheight}
    \includegraphics[width=0.9\linewidth]{centralities/karate_betweenness}
  \end{overlayarea}
\end{frame}

\begin{frame}[t]{Eigenvector centrality}
  \begin{overlayarea}{\linewidth}{5mm}
    A node is as central as its neighbors %
    \only<1>{(Step 1)}%
    \only<2>{(Step 2)}%
    \only<3>{(Step 3)}%
    \only<4>{(Step 4)}%
    \only<5>{(Final)}
  \end{overlayarea}
  \begin{overlayarea}{\linewidth}{\textheight}
    \only<1>{\includegraphics[width=0.9\linewidth]{centralities/karate_eigenvector_0}}
    \only<2>{\includegraphics[width=0.9\linewidth]{centralities/karate_eigenvector_1}}
    \only<3>{\includegraphics[width=0.9\linewidth]{centralities/karate_eigenvector_2}}
    \only<4>{\includegraphics[width=0.9\linewidth]{centralities/karate_eigenvector_3}}
    \only<5>{\includegraphics[width=0.9\linewidth]{centralities/karate_eigenvector}}
  \end{overlayarea}
\end{frame}

\begin{frame}[t]{Pagerank}
  \begin{overlayarea}{\linewidth}{5mm}
  A node is as central as encountered in a random walk %
  \end{overlayarea}
  \begin{overlayarea}{\linewidth}{\textheight}
      \only<1>{\includegraphics[width=0.9\linewidth]{centralities/karate_pagerank_0}}
      \only<2>{\includegraphics[width=0.9\linewidth]{centralities/karate_pagerank_1}}
      \only<3>{\includegraphics[width=0.9\linewidth]{centralities/karate_pagerank_2}}
      \only<4>{\includegraphics[width=0.9\linewidth]{centralities/karate_pagerank_3}}
      \only<5>{\includegraphics[width=0.9\linewidth]{centralities/karate_pagerank_4}}
      \only<6>{\includegraphics[width=0.9\linewidth]{centralities/karate_pagerank_5}}
      \only<7>{\includegraphics[width=0.9\linewidth]{centralities/karate_pagerank_6}}
      \only<8>{\includegraphics[width=0.9\linewidth]{centralities/karate_pagerank_7}}
      \only<9>{\includegraphics[width=0.9\linewidth]{centralities/karate_pagerank_8}}
      \only<10>{\includegraphics[width=0.9\linewidth]{centralities/karate_pagerank_9}}
  \end{overlayarea}
\end{frame}


\subsection{Weak ties}
\frame[c]{\Huge \centering Weak ties}

\begin{frame}[c]{Weak ties}
  \begin{itemize}
    \item Weak ties are links that have a low weight.
    \item Ties are as strong as their overlap.
    \item Weak ties hold together the network.
    \item Strong ties are clustered.
    \item New information comes in via weak ties.
  \end{itemize}
\end{frame}

\begin{frame}[t]{Weak ties}
  Weight within classes $\approx 44$, between classes $\approx 7$.
  \includegraphics[width=0.9\linewidth]{sociopatterns_class_weak_links}
\end{frame}

\begin{frame}[t]{Overlap}
  Jaccard similarity: relative number of nodes in common.

  \vfill

  \begin{center}
    2 out of 8 nodes in common $=\frac{2}{8} = 0.25$
    \begin{tikzpicture}
      \node[n] (a) at (-1, 0) {a};
      \node[n] (b) at ( 1, 0) {b};
      \draw[-,very thick] (a) edge node[midway] (ab) {} (b);

      \node[n,above left=1cm of a] (c) {};
      \node[n,left=1cm of a]       (d) {};
      \node[n,below left=1cm of a] (e) {};
      \path[-] (a) edge (c)
                   edge (d)
                   edge (e);

      \node[n,above right=1cm of b] (f) {};
      \node[n,right=1cm of b]       (g) {};
      \node[n,below right=1cm of b] (h) {};
      \path[-] (b) edge (f)
                   edge (g)
                   edge (h);

      \node[n,fill=Set1-3-2,above=1cm of ab] (i) {};
      \node[n,fill=Set1-3-2,below=1cm of ab] (j) {};
      \path[-] (a) edge (i)
                   edge (j)
               (b) edge (i)
                   edge (j);
    \end{tikzpicture}  
  \end{center}
  
\end{frame}

\begin{frame}[t]{Weak ties}
  Correlation between (log of) weight and overlap about 0.4
  \begin{center}
    \includegraphics[width=0.9\linewidth]{weight_overlap}
  \end{center}
\end{frame}

\section{Structural holes}

\begin{frame}[t]{Structural holes}
  \begin{itemize}
    \item Structural holes are the ``missing links''.

    \begin{center}
      \begin{tikzpicture}
        \node[n] (0) at (0,0) {0};
        
        \node[n] (1) at (-30:1.2cm)   {1};
        \node[n] (2) at ( 90:1.2cm) {2};
        \node[n] (3) at (210:1.2cm) {3};

        \path[-] (0) edge (1)
                     edge (2)
                     edge (3)
                 (1) edge (2);

        \draw[dashed] (150:1.1cm) circle [x radius=9mm, y radius=4mm, rotate=60];
        \draw[dashed] (-90:1.1cm) circle [x radius=9mm, y radius=4mm];
      \end{tikzpicture}  
    \end{center}
    
    \item A node which ``fills'' a structural hole is a local bridge or \emph{broker}.
    \item Low clustering $\Rightarrow$ high broker.
    \item Structural holes provide strategic advantages.
    \item Receive more and diverse information, good for creativity.
  \end{itemize}
\end{frame}

\begin{frame}[t]{Broker}
  \begin{overlayarea}{\linewidth}{5mm}
    High broker has contacts across the network.
  \end{overlayarea}
  \begin{overlayarea}{\linewidth}{\textheight}
      \centering
      \includegraphics[width=0.9\linewidth]{most_broker}
  \end{overlayarea}
\end{frame}

\begin{frame}[t]{Broker}
  \begin{overlayarea}{\linewidth}{5mm}
    Low broker has contacts more concentrated in one group. 
  \end{overlayarea}
  \begin{overlayarea}{\linewidth}{\textheight}
    \centering
    \includegraphics[width=0.9\linewidth]{least_broker}
  \end{overlayarea}
\end{frame}

\begin{frame}[t]{Broker}
  \begin{overlayarea}{\linewidth}{5mm}
    Brokerage correlates with distribution across groups.
  \end{overlayarea}
  \begin{overlayarea}{\linewidth}{\textheight}
    \centering
    \includegraphics[width=0.9\linewidth]{Brokerage}
  \end{overlayarea}
\end{frame}

\subsection{Equivalence}
\frame[c]{\Huge \centering Equivalence}

\tikzstyle{n1}=[n,fill=Set1-8-1]
\tikzstyle{n2}=[n,fill=Set1-8-2]
\tikzstyle{n3}=[n,fill=Set1-8-3]
\tikzstyle{n4}=[n,fill=Set1-8-4]
\tikzstyle{n5}=[n,fill=Set1-8-5]
\tikzstyle{n6}=[n,fill=Set1-8-6]
\tikzstyle{n7}=[n,fill=Set1-8-7]

\begin{frame}[c]{Equivalence}
  \begin{columns}
    \begin{column}{0.7\textwidth}
     \begin{itemize}
        \item Some nodes are in some sense equal.
        \item 
          \only<1,3->{Structural equivalence}
          \only<2>{\textbf{Structural equivalence}}
          \begin{itemize}
            \item Share the same neighbors.
          \end{itemize}
        \item 
          \only<1-2,4->{Isomorphic equivalence}
          \only<3>{\textbf{Isomorphic equivalence}}
          \begin{itemize}
            \item Labelling of nodes can be switched.
          \end{itemize}
        \item 
          \only<1-3,5->{Regular equivalence}
          \only<4>{\textbf{Regular equivalence}}
          \begin{itemize}
            \item Two nodes share same type of neighbors.
            \item Sometimes called ``role detection''.
          \end{itemize}
      \end{itemize}   
    \end{column}
    \begin{column}{0.3\textwidth}
      \begin{tikzpicture}
        \only<1>{
          \node[n] {a} 
            child { node[n] {b} 
                    child { node[n] {e} }
                  } 
            child { node[n] {c} 
                    child { node[n] {f} } 
                    child { node[n] {g} } 
                  } 
            child { node[n] {d} 
                    child { node[n] {h} }
                  };
        }
        \only<2>{
          \node[n1] {a} 
            child { node[n2] {b} 
                    child { node[n3] {e} }
                  } 
            child { node[n4] {c} 
                    child { node[n5] {f} } 
                    child { node[n5] {g} } 
                  } 
            child { node[n6] {d} 
                    child { node[n7] {h} }
                  };
        }
        \only<3>{
          \node[n1] {a} 
            child { node[n2] {b} 
                    child { node[n3] {e} }
                  } 
            child { node[n4] {c} 
                    child { node[n5] {f} } 
                    child { node[n5] {g} } 
                  } 
            child { node[n2] {d} 
                    child { node[n3] {h} }
                  };
        }
        \only<4>{
          \node[n1] {a} 
            child { node[n2] {b} 
                    child { node[n3] {e} }
                  } 
            child { node[n2] {c} 
                    child { node[n3] {f} } 
                    child { node[n3] {g} } 
                  } 
            child { node[n2] {d} 
                    child { node[n3] {h} }
                  };
        }
      \end{tikzpicture}
    \end{column}
  \end{columns}
  
\end{frame}

\subsection{Social balance}
\frame[c]{\Huge \centering Social balance}

\begin{frame}
	\frametitle{Social balance}
	\only<1>{Two friends are fighting. (unstable)}
	\only<2>{All friends, everybody happy. (stable)}
	\only<3>{Two friends are fighting. (unstable)}
	\only<4>{My enemy's enemy is my friend. (stable)}
	\only<5>{Mutual enemies. (unstable)}
	\only<6>{My enemy's enemy is my friend. (stable)}
	\begin{center}
		\begin{tikzpicture}[x=5mm,y=5mm]

			\tikzset{p/.style={n,minimum size=10mm}}
			\tikzset{e/.style={very thick,Set1-6-2}}
			\tikzset{e2/.style={e,Set1-6-1}}

			\only<1>{\triad{triad1}{e}{e}{e2}{John}{Mike}{Pete}}
			\only<2>{\triad{triad1}{e}{e}{e}{John}{Mike}{Pete}}
			\only<3>{\triad{triad1}{e}{e}{e2}{John}{Mike}{Pete}}
			\only<4>{\triad{triad1}{e}{e2}{e2}{John}{Mike}{Pete}}
			\only<5>{\triad{triad1}{e2}{e2}{e2}{John}{Mike}{Pete}}
			\only<6>{\triad{triad1}{e}{e2}{e2}{John}{Mike}{Pete}}
		
		\end{tikzpicture}	
	\end{center}
\end{frame}

\begin{frame}[t]{Social balance} 

  \includegraphics[width=\textwidth]{soc_balance}

  Weak social balance: more than two factions.
\end{frame}


\section{Complex networks}
\frame[c]{\Huge \centering Complex networks}

\subsection{Random graphs}
\frame[c]{\Huge \centering Random graphs}

\begin{frame}[c]{Random graphs}

  \begin{columns}[t]

    \begin{column}{.5\textwidth}
      \begin{block}{Why}
        \begin{itemize}
          \item Null-model
          \item For testing
          \item For analysis
          \item Approximation, Model
        \end{itemize}  
      \end{block}
    \end{column}

    \begin{column}{.5\textwidth}
      \begin{block}{What}
        \begin{itemize}
          \item Erdös-Rényi graphs
          \item Small world
          \item Configuration model
          \item Scale-free model
        \end{itemize}  
      \end{block}
    \end{column}
  \end{columns}
  
\end{frame}

\begin{frame}[c]{Erdös-Rényi graphs}
  \begin{block}{Construction}
    \begin{itemize}
      \item Create empy graph with $n$ nodes.
      \item Every link has same probability $p$ of appearing.
    \end{itemize}
  \end{block}

  \begin{block}{Characteristics}
    \begin{itemize}
      \item Degree roughly $pn$ for most nodes.
      \item Random links create short-distances
      \item Low clustering
    \end{itemize}  
  \end{block}

  Useful as null-model or first approximation.
\end{frame}
\begin{frame}[c]{Small world}
  \begin{center}
    \includegraphics[width=.9\linewidth]{small_world}
  \end{center}
\end{frame}
\begin{frame}[c]{Configuration model}
  \begin{overlayarea}{\linewidth}{5mm}
    \only<1>{Degree fixed}
    \only<2>{Particular realisation}
  \end{overlayarea}
  \vskip5mm
  \begin{overlayarea}{\linewidth}{0.6\textheight}
    \begin{center}
      \begin{tikzpicture}[x=1.8cm,y=1.6cm]
        \tikzstyle{e}=[-, shorten <= 1pt, shorten >= 1pt,
          bend right=10, line width=1pt]
        \tikzstyle{h}=[-, shorten <= 4pt, shorten >= 4pt, bend right=10,
        color=white, line width=4pt] 

        \node at (0.0000325,1.9204232)  [n] (p1)   {};
        \node at (0.4550908,0.7933351)  [n] (p2)   {};
        \node at (1.083129,0.3480107)   [n] (p3)   {};
        \node at (2.2209125,1.4114824)  [n] (p4)   {};
        \node at (0.828581,1.030818)    [n] (p5)   {};
        \node at (1.4287366,1.599514)   [n] (p6)   {};
        \node at (1.8110229,2.6793186)  [n] (p7)   {};
        \node at (0.6028443,2.6458648)  [n] (p8)   {};
        \node at (-0.600547,2.9859712)  [n] (np1)  {};
        \node at (3.4355246,1.485168)   [n] (np2)  {};
        \node at (-1.4807484,0.4648599) [n] (np3)  {};
        \node at (-2.1030472,1.5738374) [n] (np4)  {};
        \node at (-0.453152,1.6447195)  [n] (np5)  {};
        \node at (1.876317,0)           [n] (np6)  {};
        \node at (2.7219952,0.7565929)  [n] (np7)  {};
        \node at (-0.7388882,0.5948308) [n] (np8)  {};
        \node at (-2.194684,0.6428498)  [n] (np9)  {};
        \node at (-1.3501299,1.4698088) [n] (np10) {};
        \node at (-1.103319,2.4284336)  [n] (np11) {};
        \node at (2.8991467,2.331418)   [n] (np12) {};
        \begin{pgfonlayer}{background}
          \only<1-2>{
            \draw[e]
              (p1)   edge (p6)
              (p1)   edge (p7)
              (p1)   edge (p8)
              (p2)   edge (p5)
              (p3)   edge (p4)
              (p4)   edge (p8)
              (p5)   edge (p6)
              (p5)   edge (p8)
              (p6)   edge (p7)
              (p6)   edge (p8)
              (p7)   edge (p8)
              (np4)  edge (np9)
              (np4)  edge (np10)
              (np5)  edge (np1)
              (np5)  edge (np3)
              (np5)  edge (np10)
              (np2)  edge (np7)
              (np2)  edge (np12)
              (np3)  edge (np8)
              (np3)  edge (np9)
              (np10) edge (np11)
              (p1)   edge (np8)
              (p2)   edge (np8)
              (p2)   edge (np10)
              (p3)   edge (np7)
              (p3)   edge (np8)
              (p4)   edge (np7)
              (p4)   edge (np12)
              (p5)   edge (np10)
              (p6)   edge (np7)
              (p8)   edge (np11)
              (p5)   edge (np5)
              (p3)   edge (np6)
              (p8)   edge (np1);
            }
          \only<1>{
            \draw[h]
              (p1)   edge (p6)
              (p1)   edge (p7)
              (p1)   edge (p8)
              (p2)   edge (p5)
              (p3)   edge (p4)
              (p4)   edge (p8)
              (p5)   edge (p6)
              (p5)   edge (p8)
              (p6)   edge (p7)
              (p6)   edge (p8)
              (p7)   edge (p8)
              (np4)  edge (np9)
              (np4)  edge (np10)
              (np5)  edge (np1)
              (np5)  edge (np3)
              (np5)  edge (np10)
              (np2)  edge (np7)
              (np2)  edge (np12)
              (np3)  edge (np8)
              (np3)  edge (np9)
              (np10) edge (np11)
              (p1)   edge (np8)
              (p2)   edge (np8)
              (p2)   edge (np10)
              (p3)   edge (np7)
              (p3)   edge (np8)
              (p4)   edge (np7)
              (p4)   edge (np12)
              (p5)   edge (np10)
              (p6)   edge (np7)
              (p8)   edge (np11)
              (p5)   edge (np5)
              (p3)   edge (np6)
              (p8)   edge (np1);
          }

        \end{pgfonlayer}
      \end{tikzpicture}
    \end{center}
  \end{overlayarea}
\end{frame}
\begin{frame}
  \frametitle{Degree distribution}

  \begin{tikzpicture}
    \only<1>{
      \begin{axis}[
          domain=1:100,
          no markers,
          xlabel=Degree,
          ylabel=Probability,
          name=plot,
          width=0.8\textwidth,
          height=0.6\textheight,
          scale only axis,
          yticklabels={,,},
          ymax=0.2]
        \addplot[color=Set1-6-2,thick,samples=200,smooth]
          {x^(-2)}
          node[pos=0.02,pin=right:{Scale free}] {};
        \addplot[color=Set1-6-3,thick,samples=200,smooth]
          gnuplot[id=poisson1] {15^x*exp(-15)/gamma(x+1)}
          node[pos=0.2,pin=above right:{Erdös-Rényi}] {};
      \end{axis}
    }
    \only<2>{
      \begin{loglogaxis}[
          domain=1:100,
          no markers,
          xlabel=Degree,
          ylabel=Probability,
          name=plot,
          width=0.8\textwidth,
          height=0.6\textheight,
          scale only axis,
          yticklabels={,,},
          ymin=10e-8]
        \addplot[color=Set1-6-2,thick,samples=200,smooth]
          {x^(-2)}
          node[pos=1,pin=above:{Hubs}] {};
        \addplot[color=Set1-6-3,thick,samples=200,smooth]
          gnuplot[id=poisson2] {15^x*exp(-15)/gamma(x+1)};
      \end{loglogaxis}
    }
  \end{tikzpicture}

  \begin{itemize}
    \item In many real networks, scale free degree distribution. 
    \item In ER graphs, Poisson degree distribution. 
  \end{itemize}

\end{frame}
\begin{frame}[c]{Scale-free}

  How to get scale-free networks?

  \vskip1cm

  \begin{block}{Construction}
    Start with graph with $q$ nodes
    \begin{enumerate}
      \item Add node
      \item Add $q$ links with probability proportional to degree.
      \item Repeat (1)-(2).
    \end{enumerate}
  \end{block}

  \begin{block}{Preferential attachment}
    A node that already has many links has higher probability to receive yet
    another link.
  \end{block}

  Also known as cumulative advantage, rich-get-richer, or Matthew effect.

\end{frame}

\subsection{Spreading}
\frame[c]{\Huge \centering Spreading}

\begin{frame}[c]{Spreading}

  \begin{block}{Spreading}
    \begin{itemize}
      \item Already briefly encountered as social influence
      \item Here, focus on simple contagion.
      \item Biggest question: will it spread?
        \begin{itemize}
          \item Giant component
          \item Random failure
        \end{itemize}
    \end{itemize}  
  \end{block}

  \vskip1cm

  \begin{block}{Scale free}
    \begin{itemize}
      \item Scale free networks robust again random node failure.
      \item Vulnerable for targeted attacks (take out the hubs).
      \item No threshold for spreading.
    \end{itemize}  
  \end{block}
  
\end{frame}

\begin{frame}[c]{Spreading}
    \includegraphics[width=\linewidth]{infection}
\end{frame}

\subsection{Community detection}
\frame[c]{\Huge \centering Community detection}

\tikzstyle{p}=[line width=6mm, line cap=round, line join=round]
\tikzstyle{e}=[color=black,bend left=15]
\tikzstyle{es}=[opacity=0.2,bend left=15,dashed]
\tikzstyle{n}=[font=\tiny,shape=circle,draw=Set1-8-1,fill=Set1-8-1!50, thick,
               minimum size=4mm, outer sep=2pt, inner sep=1pt]
\tikzstyle{n0}=[n,draw=Set1-8-1,fill=Set1-8-1!50]
\tikzstyle{n1}=[n,draw=Set1-8-2,fill=Set1-8-2!50]
\tikzstyle{n2}=[n,draw=Set1-8-3,fill=Set1-8-3!50]
\tikzstyle{n3}=[n,draw=Set1-8-4,fill=Set1-8-4!50]
\tikzstyle{n4}=[n,draw=Set1-8-5,fill=Set1-8-5!50]
\tikzstyle{n5}=[n,draw=Set1-8-6,fill=Set1-8-6!50]
\tikzstyle{n6}=[n,draw=Set1-8-7,fill=Set1-8-7!50]
\tikzstyle{n7}=[n,draw=Set1-8-8,fill=Set1-8-8!50]
\tikzstyle{n8}=[n,draw=Set2-8-1,fill=Set2-8-1!50]
\tikzstyle{bb}=[rectangle,outer sep=0pt, inner sep=0pt]
\tikzstyle{l}=[font=\large\sffamily\bfseries]
\tikzstyle{img}=[inner sep=0, outer sep=0]
\tikzstyle{s}=[pencil,draw,thick,red,rectangle]
\tikzstyle{s2}=[pencil small,draw,thick,red,rectangle]

\tikzstyle{notepaper}=[pencil,Set1-8-2!20,thin,opacity=0.5]
\tikzstyle{noteborder}=[pencil,Set1-8-2,thick]

\DTLloaddb[noheader=false]{nodes}{data/nodes.dat}
\DTLloaddb[noheader=false]{links}{data/links.dat}

\begin{frame}
  \frametitle{What is a community?}

  \begin{tikzpicture}
    \begin{pgfonlayer}{background}
      \path (-6,-4) rectangle  (6,4);
    \end{pgfonlayer}
    \begin{scope}[x=1.8cm,y=1.8cm,xshift=1.5cm, yshift=-1.5cm]
    % Draw nodes
      \DTLforeach*{nodes}{\n=node, \x=x, \y=y, \comm=commbad}
      { \node[n\comm] (\n) at (\x,\y) {\n}; }
      % Draw nodes
      \DTLforeach*{links}{\from=from,\to=to}
      { \draw (\from) edge[e] node[img] (\from-\to) {} (\to); }
    \end{scope}
    \node[bb,fit=(0) (1) (2) (3) (4) (5) (6) (7) (8) (9) (10) (11)] (G) {};
    \begin{pgfonlayer}{background}
      \draw[p,color=Set1-8-1!30, fill=Set1-8-1!30] 
      plot [smooth cycle] 
      coordinates {(1) (2) (11)};

      \draw[p,color=Set1-8-2!30, fill=Set1-8-2!30] 
      plot [smooth cycle] 
      coordinates {(5) (9) (8) (10) (7)};

      \draw[p,color=Set1-8-3!30, fill=Set1-8-3!30] 
      plot [smooth cycle] 
      coordinates {(3) (0) (4) (6)};
    \end{pgfonlayer}

    \only<1-8>
    {
      \draw[red,thick,smooth] \convexpath{3,0,4,6}{4mm};
      \node[left=1cm of 4,yshift=3cm] (desc) {\textbf{Community}};
      \draw[shorten <=1.5mm] (6) edge[s,bend right=15] (desc.east);
    }
    \only<2-8>{
      \node[below=2mm of desc,anchor=east] (good) {But is it a \textbf{good} community?};
    }
    \only<3-8>{
      \draw[pencil,Set1-8-2,thick] (-4.5,-2.3) rectangle ($(4)+(-1,1.8)$);
      \draw[pencil,Set1-8-2!20,thin,opacity=0.5] (-4.5,-2.3)
      grid[xstep=2mm,ystep=2mm] ($(4)+(-1,1.8)$);
    }
    \only<3-8>{
      \node[anchor=west] at (-4.3, 2.1) (count) {Count links in community};
    }
    \only<4-8>{
      \node[below=5mm of count.west,anchor=west] (links) {~$e_c = 4$};
    }
    \only<5-8>{
      \node[below=8mm of links.west,anchor=west] (commsize) {Expected links?};
    }
    \only<6-8>{
      \node[below=6mm of commsize.west,anchor=west] (posslinks) 
      {~$\frac{1}{2}\frac{K_c^2}{2m} = \frac{1}{2}\frac{12^2}{2 \cdot 17} \approx 2.12$};
    }
    \only<7-8>{
      \node[below=9mm of posslinks.west,anchor=west] (density_txt) {};
      \node[below=6mm of density_txt.west,anchor=west, text width=4cm] (density) 
      {$4$ links in community,\\ $2.12$ expected.};
    }
    \only<8-8>{
      \node[s,fit=(density)] (dens_circle) {};
      \node[below right=7mm of density, yshift=-1mm] (dens_desc) 
      {Want many links, few expected};
      \draw[s] (dens_desc) -- (dens_circle);
    }
  \end{tikzpicture}
\end{frame}

\newcounter{d}
\begin{frame}
  \frametitle{Resolution limit}

  \begin{center}
    \begin{tikzpicture}
    %\draw[black] (-3, -4) grid[step=1] (8,4);
    %\draw[white] (-3, -4) rectangle (8,4);
      \tikzstyle{e}=[semitransparent, very thin]
      \tikzstyle{cover}=[rectangle, rounded corners,fill=Set1-3-2,nearly transparent,inner sep=0.1cm, outer sep=0cm]
      \tikzstyle{n}=[circle,draw=Set1-3-2!50,fill=Set1-3-2!50,outer sep=0mm, inner sep=0mm]

    % Parameters
      \def\rc{2cm}
      \def\nc{12}
      \def\n{7}
      \def\r{0.35cm}
    % Internal variables
      \setcounter{d}{0}
      \foreach \c in {1,...,\nc}
      {
        \clique{\c}{\n}{\r}{(\c / \nc * 360 - 45:\rc)}{130 - 360/\n - 1/\nc + \c/ \nc * 360}{1mm}
        \ifnum\c>1
        \draw[e,-] (\c-1) -- (\arabic{d}-1);
        \fi
        \begin{pgfonlayer}{background}
          \node[circle,fill=Set1-3-2!20,minimum size=2*\r+2mm, outer sep=0mm, inner
          sep=0mm] (\c) at (\c / \nc * 360 - 45:\rc) {};
        \end{pgfonlayer}
        \addtocounter{d}{1}
      }

      \only<2->{
        \setcounter{d}{1}
        \draw[-,e] (\nc-1) -- (\arabic{d}-1);
        \setcounter{d}{2}
        \begin{pgfonlayer}{background}

        %\node (comm-1-2) [cover, fit=(1) (2)]  {};
          \foreach \c in {1,3,...,\nc}
          {
            \draw[line width=2*\r+2mm,draw=Set1-3-2!20,line cap=rect] (\c) to (\arabic{d});
            \addtocounter{d}{2}
          }
        \end{pgfonlayer}
      }
    \end{tikzpicture}
  \end{center}

  \only<3->{
    \begin{tikzpicture}[remember picture, overlay]
      \fill[white, opacity=0.85] 
        (current page.south west) rectangle 
        ($(current page.north east)-(0,10mm)$);
      \node[minimum width=9cm, minimum height=7cm, 
        inner sep=0, outer sep=0, text centered,
        line width=7mm, black, text width=\textwidth] at (current page.center)
        {Modularity with resolution parameter};
    \end{tikzpicture}
  }
\end{frame}

\begin{frame}[label=upper_resolution]
  \frametitle{Upper resolution limit}
  \includegraphics[width=\textwidth]{upper_limit} 
  \vskip1cm
  \begin{center}
    \begin{tikzpicture}
      \draw[->] (-1,0) -- node[below] {Higher resolution parameter} (9,0);
    \end{tikzpicture}
  \end{center}
\end{frame}
\begin{frame}
  \frametitle{Community Detection}

  \begin{block}{Resolution limit free}
    \begin{itemize}
      \item Only few methods have no resolution limit.
      \item Constant Potts Model (CPM) one of few such methods.
      \item You have to choose a resolution $\gamma$.
      \begin{itemize}
        \item Internal density $p_c > \gamma$.
        \item Density between $p_{cd} < \gamma$.
      \end{itemize}
    \end{itemize}
  \end{block}
  \begin{tikzpicture}[remember picture, overlay]
    \only<2->
    {
      \fill[fill=white, opacity=0.9] 
        (current page.north west) ++(0,-10mm) rectangle 
        (current page.south east);
      \node[font=\Huge] at (current page.center) {How to choose $\gamma$?};
    }
  \end{tikzpicture}
\end{frame}

\begin{frame}[c]{Resolution profile}
  \includegraphics[width=0.9\linewidth]{scanres}
\end{frame}

\frame[c]{\Huge \centering Recap}

\begin{frame}[c]{Recap}

  \begin{columns}[t]

    \begin{column}{.5\textwidth}
      \begin{block}{Social network analysis}
        \begin{itemize}
          \item Homophily
          \item Social influence
          \item Weak ties
          \item Structural holes
          \item Equivalence
          \item Social balance
        \end{itemize}
      \end{block}
    \end{column}

    \begin{column}{.5\textwidth}
      \begin{block}{Complex networks}
        \begin{itemize}
          \item Random graphs 
          \item Spreading
          \item Community detection
        \end{itemize}
      \end{block}
    \end{column}

  \end{columns}

\end{frame}

\end{document}
